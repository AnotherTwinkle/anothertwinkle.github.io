\documentclass{article}

\usepackage{listings, color}
\usepackage{graphicx, amsmath, amssymb, amsthm}

\newcommand{\linl}[1]{\lstinline{#1}}

% For all the code stuff
\definecolor{dkgreen}{rgb}{0,0.6,0}
\definecolor{gray}{rgb}{0.5,0.5,0.5}
\definecolor{mauve}{rgb}{0.58,0,0.82}

\lstset{frame=tb,
  language=c,
  aboveskip=3mm,
  belowskip=3mm,
  showstringspaces=false,
  columns=flexible,
  basicstyle={\small\ttfamily},
  numbers=none,
  numberstyle=\tiny\color{gray},
  keywordstyle=\color{blue},
  commentstyle=\color{dkgreen},
  stringstyle=\color{mauve},
  breaklines=true,
  breakatwhitespace=true,
  tabsize=3
}

\title{Day 1 Assignment- C specific I/O}
\author{Chotokhato Programming Camp 2024}

\begin{document}
\maketitle

\section*{Input/Output Reference}

\begin{itemize}
  \item \verb|%d| - Integer (\verb|int|) \lstinline{int a; scanf("%d", &a);}
  
  \item \verb|%f| - Floating Point Integer (\verb|float|) \lstinline{float a; scanf("%f", &a);}

  \item \verb|%lf| - Double (\verb|double|) \lstinline{double a; scnaf("%lf", &a);}

  \item \verb|%c| - \textbf{One} Character (\verb|char|) \lstinline{char a; scanf("%c", &a);}

  \item \verb|%10c| - \textbf{10} Characters (\verb|char[]|) \lstinline{char a[10]; scanf("%10c", a);}

  \item \verb|%s| - A string (\verb|char[]|) \lstinline{char a[10]; scanf("%s", a);}
\end{itemize}

\subsection*{Examples}
You can ofcourse, combine these delimiters.

\begin{lstlisting}
#include <stdio.h>

int main() {
  // Say, input is : 10 power 1.2

  int first;
  char word[5];
  float second;

  scanf("%d %s %f", &first, word, &second); // You don't need & for arrays.

  printf("Integer : %d\nWord : %s\nFloat : %f\n", first, word, second);
  return 0;

}
\end{lstlisting}

\section*{Assignments}


\end{document}
