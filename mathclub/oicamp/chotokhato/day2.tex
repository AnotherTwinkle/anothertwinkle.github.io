\documentclass{article}

\usepackage{listings, color, hyperref}
\usepackage{graphicx, amsmath, amssymb, amsthm}

\newcommand{\linl}[1]{\lstinline{#1}}

% For all the code stuff
\definecolor{dkgreen}{rgb}{0,0.6,0}
\definecolor{gray}{rgb}{0.5,0.5,0.5}
\definecolor{mauve}{rgb}{0.58,0,0.82}

\lstset{frame=tb,
  language=c,
  aboveskip=3mm,
  belowskip=3mm,
  showstringspaces=false,
  columns=flexible,
  basicstyle={\small\ttfamily},
  numbers=none,
  numberstyle=\tiny\color{gray},
  keywordstyle=\color{blue},
  commentstyle=\color{dkgreen},
  stringstyle=\color{mauve},
  breaklines=true,
  breakatwhitespace=true,
  tabsize=3
}

\title{Assignment 2}
\author{Chotokhato Programming Camp 2024}

\begin{document}
\maketitle

Please submit your codes in the following google form - \url{https://forms.gle/bGfFdVE4EhMPnMGb8}.


\section*{Task 1 - A mathematician's dream}
Given two \verb|float|s $a$ and $b$, Compute $(a+b)^2$ and $(a-b)^2$.

\section*{Task 2 - In our prime}
You are given a positive integer $x$ that is \textbf{not greater than 10} (i.e $x \le 10$). Determine if $x$ is a prime or not.

A prime number is a number that is only divisible by itself and $1$. $1$ is not a prime number.

\section*{Task 3 - Leaping through years} 
A \textit{leap year} is a year that is \textbf{divisible by 4}, but \textbf{not divisible by 100}. However, years that are \textbf{divisible by 400} are considered to be leap years.

Given an year as input, print if it is a leap year or not.

\section*{Task 4 - Second by second}
You have given an amount in seconds. You have to determine this amount in \verb|a hours b minutes c seconds| format. Example input/output is given below-

\begin{lstlisting}
input : 3600
output : 1 hours 0 minutes 0 seconds

input : 2443
output : 0 hours 40 minutes 43 seconds

input : 25
output : 0 hours 0 minutes 25 seconds
\end{lstlisting}

\end{document}

2 3 5 7 11 13