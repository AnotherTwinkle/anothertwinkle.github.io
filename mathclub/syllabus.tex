%%%%%%%%%%%%%%%%%%%%%%%%%%%%%%%%%%%%%%%%%%%%%%%%%%%%%%%%%%%%%%%
%
% Welcome to Overleaf --- just edit your LaTeX on the left,
% and we'll compile it for you on the right. If you open the
% 'Share' menu, you can invite other users to edit at the same
% time. See www.overleaf.com/learn for more info. Enjoy!
%
%%%%%%%%%%%%%%%%%%%%%%%%%%%%%%%%%%%%%%%%%%%%%%%%%%%%%%%%%%%%%%%
\documentclass[oneside]{book}
\usepackage{amsmath}
\usepackage{titlesec}

\pagestyle{myheadings}
\titleformat{\chapter}[display]
  {\normalfont\bfseries}{}{0pt}{\Huge}

\title{Dinajpur Math Club}
\author{Syllabus 2023}
\date{Final Revision}

\begin{document}
\maketitle

\chapter{Preamble}
\section{What this is}
    This is a Syllabus, Topic List or even a loosely organized class plan for Dinajpur Math Club's 2023 operations. \newline
    The topic have been divided into three \emph{ranks} (Gauss, Fermat and Euler) based on the members' general position in the academic ladder and even more specifically, their level of understanding of Olympiad Mathematics.
    \subsection{What this contains}
    This syllabus is sourced from multiple sources. Namely, for each rank,
    \begin{itemize}
        \item What academic knowledge is expected
        \item What mathematical knowledge is expected to secure a tough competitive attitude from BdMO regionals
        \item What knowledge fun (and important!) problems require in general
    \end{itemize}
    
\section{What this is \emph{not}}
    This list is not \emph{exhaustive.} This does not cover absolutely all required or important topics of competitive mathematics. \textbf{This list should not be followed strictly.} But rather, as a skeleton of what should be discussed. The topics mentioned are \textbf{not} organized by importance, dependency, or time either. Some parts of this list are left incomplete and are expected to be completed when we explore those topics in class.
    
\section{Who made this?}
    We too, just like you- are members of this club! May the second differential of everyone's life be negative :)

\chapter{Gauss}
\section{Introduction}
    Gauss focuses on Primary level students completely new to Olympiad mathematics.

\section{Barebones of Problem Solving}
Riddles! We put extra focus to develop a solver's attitude among the new intakes.
\begin{itemize}
    \item We teach how to-
     \begin{enumerate}
        \item Understand statements
        \item Approach a problem
        \item Come up with vague ideas and conjectures
        \item Find patterns
        \item Actually come up with a rigorous solution
     \end{enumerate}
     
    \item Brain Puzzles
     \begin{itemize}
        \item The early classes begin with interesting problems, that don't need much mathematical  knowledge, if at all. \textbf{Sourcing: BdMO Regional P1-P3}.
     \end{itemize}
    
    \item Arithmetic Puzzles
     \begin{itemize}
         \item The aforementioned puzzles should slowly turn into problems involving (some form of) mathematical intuition. Say, for example- The Census Taker problem. \textbf{Sourcing: BdMO Regional P1-P5}.
     \end{itemize}
\end{itemize}

\section{Number Theory}
\begin{itemize}
  \item Numbers
    \begin{enumerate}
        \item Positive and negative integers
        \item Understanding of signed operations (i.e $-(-5) = 5$)
        \item Fractions, Rational and Irrational numbers
    \end{enumerate}
  \item Divisibility
    \begin{enumerate}
      \item Integer Divisibility tricks (Divisibility criteria for 2, 3, 5 etc.)
      \item Parity
  \end{enumerate}
  \item Primes and Factors
    \begin{enumerate}
      \item Introduction to primes
      \item Factorization of integers
      \item G.C.D and L.C.M
  \end{enumerate}
\end{itemize}

\section{Counting}
\begin{itemize}
    \item Theory
        \begin{enumerate}
            \item Addition and Multiplication principle
            \item Factorials (and why to use them)
            \item Arrangements
            \item $C^{n}_r$ and $P^{n}_r$ (Note: This is actually rather advanced)
        \end{enumerate}
    \item Problem Types
        \begin{enumerate}
            \item Bruteforce solutions
            \item Counting puzzles
            \item Ways to choose $n$ things from $m$ things such that...
            \item Ways to arrange $n$ people such that...
            \item Ways to construct a word with $N$ letters such that...
            \item Ways to travel from $A_1 \rightarrow A_2\rightarrow ... \rightarrow A_{n-1} \rightarrow A_n$
            \item Find number of $n$ digit integers such that...
            \item Round-table versions of all arrangement problems
            \item Palindrome related problems (Basics)
        \end{enumerate}
\end{itemize}

 \section{Geometry}
 \begin{itemize}
     \item Triangles
        \begin{enumerate}
            \item Properties of Equilateral, Isosceles and Right triangles
            \item Theorems related to area and side lengths
            \item Congruence ($SAS$, $SSS$ etc)
            \item Similarity and ratios
            \item Properties of angles of a triangle
            \item Pythagorean Theorem
        \end{enumerate}
    \item Quadrilaterals
        \begin{enumerate}
            \item Properties of Squares, Parallelograms, Trapezoids etc
            \item Theorems related to area and diagonals
        \end{enumerate}
    \item Circles
        \begin{enumerate}
            \item Theorems related to area and ratio of circles
            \item Properties of circumference, radius, segments, arcs etc.
        \end{enumerate}
    \item Basic Angle and Segment chasing
    \item Properties of different types of Polygons
 \end{itemize}
 
 \section{Algebra}
 Note that class 6-7 Algebra suffices.
\begin{itemize}
    \item Variables, Exponents, Algebraic manipulation etc.
    \item Solving basic equations
    \item Square and Cubic formulas (i.e $(a+b)^2 = a^2 + 2ab + b^2$ etc.)
 \end{itemize}


\chapter{Fermat}
\section{Introduction}
    Fermat focuses on students around advanced primary, beginner and intermediate junior level.
    Just like Gauss, we dedicate the first few classes to skim away from mathematics and focus on Problem Solving in general. (See \emph{Barebones of Problem Solving}).
    
\section{Number Theory}
\begin{itemize}
  \item Divisibility
    \begin{enumerate}
      \item Integer Divisibility tricks (Divisibility criteria for 2, 3, 5 etc.)
      \item Basic Theorems (i.e $a \mid b, b \mid c \implies a \mid c$ etc.)
      \item Parity
  \end{enumerate}
  \item Primes and Factors
    \begin{enumerate}
      \item Basic prime properties
      \item Fundamental Theorem of Arithmetic
      \item Number of divisors
  \end{enumerate}
  \item Advance usage of G.C.D and L.C.M
  \item Euclidean G.C.D algorithm
  \item Factorials (i.e Largest power of $p$ that divides $n!$)
  \item Floor and ceiling functions
  \item Modular Arithmetic (This applies to Intermediate Juniors)
    \begin{enumerate}
      \item Introduction to modular notation and residue classes
      \item Basic principles
      \item Last $n$ digits problems
      \item Applying Modular tricks to equations [i.e $x^2 \equiv 0 \text{ or } 1 \hspace{1mm}(\text{mod } 4)$]
  \end{enumerate}
\end{itemize}

\newpage

\section{Counting}
This section will mainly focus on listing common problem types. Note that this list is not exhaustive.
\begin{itemize}
    \item Theory
        \begin{enumerate}
            \item Addition and Multiplication principle
            \item Factorials (and why to use them)
            \item Subsets and Arrangements
            \item $C^{n}_r$ and $P^{n}_r$
            \item Basic Bijection Principle
            \item Pigeonhole Principle
            \item Coming up with combinatorial arguments
        \end{enumerate}
    \item Problem Types
        \begin{enumerate}
            \item Bruteforce solutions
            \item Counting puzzles
            \item Ways to choose $n$ things from $m$ things such that...
            \item Ways to arrange $n$ people such that...
            \item Ways to construct a word with $N$ letters such that...
            \item Ways to travel from $A_1 \rightarrow A_2\rightarrow ... \rightarrow A_{n-1} \rightarrow A_n$
            \item Ways to distribute $n$ things among $m$ people such that... (Distribution problems)
            \item Find number of $n$ digit integers such that...
            \item Find sum of the digits of an sequence $A_1A_2A_3...A_n$ such that...
            \item Find number of paths in a grid from $A$ to $B$ such that...
            \item Find number of positive solutions to the equation $ax + by = n$...
            \item Find number of diagonals of a polygon with $n$ vertices such that...
            \item Find number of subsets of set $S$ such that...
            \item Round-table versions of all arrangement problems
            \item Palindrome related problems
            \item Chessboard problems
            \item Game problems
        \end{enumerate}
\end{itemize}
\newpage

 \section{Geometry}
 \begin{itemize}
     \item Triangles
        \begin{enumerate}
            \item Properties of Equilateral, Isosceles and Right triangles
            \item Theorems related to area and side lengths
            \item Congruence ($SAS$, $SSS$ etc)
            \item Similarity and ratios
            \item Properties of angles of a triangle
            \item Pythagorean Theorem
            \item Cyclic triangles
            \item Circumcenter, Orthocenter, Incenter etc.
            \item Basic Trigonometry properties
        \end{enumerate}
    \item Quadrilaterals
        \begin{enumerate}
            \item Properties of Squares, Parallelograms, Trapezoids etc
            \item Theorems related to area and diagonals
            \item Cyclic quadrilaterals
        \end{enumerate}
    \item Circles
        \begin{enumerate}
            \item Theorems related to area and ratio of circles
            \item Properties of circumference, radius, segments, arcs etc.
            \item Sectors and angles inscribed in a circle
            \item Power of point
        \end{enumerate}
    \item Angle and Segment chasing
    \item Properties of Geometric figures (Polygons, 3D shapes etc.)
 \end{itemize}
 
 \newpage
 
 \section{Algebra}
Students are expected to have basic and practical knowledge about class 6, 7 and 8 algebra.\newline We can observe that in this level, olympiad algebra doesn't deviate much from textbook algebra, if at all.
\begin{itemize}
    \item Pre-olympiad Algebra
        \begin{enumerate}
            \item Variables, Exponents, Algebraic manipulation etc.
            \item Solving equations
            \item Square and Cubic formulas (i.e $(a+b)^2 = a^2 + 2ab + b^2$ etc.)
            \item Knowledge about the cartesian plane
            \item Basic set theory
            \item Correlating Algebra with Geometry
        \end{enumerate}
    \item Basics about inequalities
    \item Basics about functions
    \item Series and sequences
    \item Factorization
 \end{itemize}

\chapter{Euler}
\section{Introduction}
    Euler focuses on students around advanced junior, and secondary level.
    All topics from previous ranks apply. Please note that, these topics are expected to be studied in group study sessions. Euler doesn't have a regular mentor strategy. \newline
    \textbf{Note:} Some topics only apply to national level and further. Regional topics should be prioritized.
    
\section{Number Theory}
\begin{itemize}
    \item Divisibility
    \begin{enumerate}
      \item Integer Divisibility tricks (Divisibility criteria for 2, 3, 5 etc.)
      \item All well known theorems (i.e $a \mid b, b \mid c \implies a \mid c$ etc.)
      \item Parity
  \end{enumerate}
  \item Primes and Factors
    \begin{enumerate}
      \item Well known Prime properties
      \item Fermat's Little Theorem (and prime-ness checks in general)
      \item Fermat numbers
      \item Coprimes and Euler's Totient Function
      \item Fundamental Theorem of Arithmetic
      \item Number of divisors
      \item Sum of divisors
  \end{enumerate}
  \item Advance usage of G.C.D and L.C.M
  \item Euclidean G.C.D algorithm
  \item Factorials (i.e Largest power of $p$ that divides $n!$)
  \item Floor and ceiling functions
  \item Numerical systems (Binary, Ternary etc including generalizations)
  \item Modular Arithmetic
    \begin{enumerate}
      \item Advanced usage of modular notation and residue classes
      \item Advanced principles
      \item Last $n$ digits problems
      \item Quadratic residues and contradictions
  \end{enumerate}
  \item Diophantine Equations [Note: This section is incomplete]
    \begin{enumerate}
        \item Introduction to Diophantine equations
        \item Application and proof techniques (i.e Modular Contradiction)
        \item Linear and non-linear Diophantine equations
    \end{enumerate}
\end{itemize}

\vspace{10pt}

\section{Counting}
For problem types, see Fermat.
\begin{itemize}
    \item Addition and Multiplication principle
    \item Factorials (and why to use them)
    \item Subsets and Arrangements
    \item $C^{n}_r$ and $P^{n}_r$
    \item Proof of various well known identities of the above
    \item Advanced application of Bijection Principle
    \item Pigeonhole Principle
    \item Extremal Principle and worst cases
    \item Inclusion-Exclusion Principle
    \item Counting related Graph Theory
    \item Fubini's Theorem
    \item Recurrence relations
    \item Two way counting
    \item Binomial Theorem
    \item Pascal's Triangle
    \item Probability
    \item Coming up with combinatorial arguments
\end{itemize}
\newpage

 \section{Geometry}
This section is incomplete and should be extended later on.
 \begin{itemize}
     \item Triangles
        \begin{enumerate}
            \item Properties of Equilateral, Isosceles and Right triangles
            \item Theorems related to area and side lengths
            \item Congruence ($SAS$, $SSS$ etc)
            \item Similarity and ratios
            \item Properties of angles of a triangle
            \item Pythagorean Theorem
            \item Cyclic triangles
            \item Circumcenter, Orthocenter, Incenter etc.
            \item Trigonometric properties
        \end{enumerate}
    \item Quadrilaterals
        \begin{enumerate}
            \item Properties of Squares, Parallelograms, Trapezoids etc
            \item Theorems related to area and diagonals
            \item Cyclic quadrilaterals
        \end{enumerate}
    \item Circles
        \begin{enumerate}
            \item Theorems related to area and ratio of circles
            \item Properties of circumference, radius, segments, arcs etc.
            \item Sectors and angles inscribed in a circle
            \item Power of point
        \end{enumerate}
    \item Angle and Segment chasing
    \item Properties of Geometric figures (Polygons, 3D shapes etc.)
 \end{itemize}
 
 \newpage
 
 \section{Algebra}
Students are expected to have basic and practical knowledge about class 9-10 algebra.
\begin{itemize}
    \item Pre-olympiad Algebra
        \begin{enumerate}
            \item Variables, Exponents, Algebraic manipulation etc.
            \item Solving equations
            \item Square and Cubic formulas (i.e $(a+b)^2 = a^2 + 2ab + b^2$ etc.)
            \item Difference of squares, and other well known formulas and proofs
            \item Knowledge about the Cartesian and Polar plane
            \item Set theory
            \item Correlating Algebra with Geometry
        \end{enumerate}
    \item Inequalities
    \item Functions
    \item Polynomials
    \item Series and sequences
    \item Quadratic equations
 \end{itemize}

 \section{Proofs}
 This is a special section for Euler. Here we devise various proof techniques and principles important for BdMO nationals.
 \begin{itemize}
     \item Logic
     \item Understanding statements (If and only if, For all, There exists etc)
     \item Converse, Inverse and Contraposition
     \item Contradiction
     \item Induction
     \item Problems about existence
     \item Pigeonhole, Extremal and Inclusion-Exclusion principles
 \end{itemize}
\vspace{25pt}
Final Revision - \today
\end{document}